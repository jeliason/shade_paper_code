\documentclass[12pt]{article}
\usepackage{geometry}
\usepackage{amsmath}
\usepackage{amsfonts}
\usepackage{hyperref}
\usepackage{natbib}

\geometry{margin=1in}
\usepackage{xcolor}

\newcommand{\bolds}[1]{\boldsymbol{#1}}


% boldcal
\newcommand{\bcA}{\bolds{{\cal A}}}
\newcommand{\bcH}{\bolds{{\cal H}}}
\newcommand{\bcR}{\bolds{{\cal R}}}
\newcommand{\bcZ}{\bolds{{\cal Z}}}

% cal
\newcommand{\calA}{{\cal A}}
\newcommand{\calD}{{\cal D}}
\newcommand{\calG}{{\cal G}}
\newcommand{\calI}{{\cal I}}
\newcommand{\calK}{{\cal K}}
\newcommand{\calL}{{\cal L}}
\newcommand{\calM}{{\cal M}}
\newcommand{\calP}{{\cal P}}
\newcommand{\calQ}{{\cal Q}}
\newcommand{\calS}{{\cal S}}
\newcommand{\calT}{{\cal T}}
\newcommand{\calU}{{\cal U}}
\newcommand{\calV}{{\cal V}}
\newcommand{\calX}{{\cal X}}
\newcommand{\calW}{{\cal W}}
\newcommand{\calY}{{\cal Y}}
\newcommand{\Cov}{\bolds{C}}

% bold
\newcommand{\ba}{\bolds{a}}
\newcommand{\bA}{\bolds{A}}
\newcommand{\bb}{\bolds{b}}
\newcommand{\bB}{\bolds{B}}
\newcommand{\bc}{\bolds{c}}
\newcommand{\bC}{\bolds{C}}
\newcommand{\bd}{\bolds{d}}
\newcommand{\bD}{\bolds{D}}
%\newcommand{\be}{\bolds{e}}
\newcommand{\bE}{\bolds{E}}
\newcommand{\bbf}{\bolds{f}}
\newcommand{\bF}{\bolds{F}}
\newcommand{\bg}{\bolds{g}}
\newcommand{\bG}{\bolds{G}}
\newcommand{\bh}{\bolds{h}}
\newcommand{\bH}{\bolds{H}}
\newcommand{\bi}{\bolds{i}}
\newcommand{\bI}{\bolds{I}}
\newcommand{\bJ}{\bolds{J}}
\newcommand{\bk}{\bolds{k}}
\newcommand{\bK}{\bolds{K}}
\newcommand{\bl}{\bolds{\ell}}
\newcommand{\bL}{\bolds{L}}
\newcommand{\bm}{\bolds{m}}
\newcommand{\bM}{\bolds{M}}
\newcommand{\bn}{\bolds{N}}
\newcommand{\bN}{\bolds{N}}
\newcommand{\bO}{\bolds{O}}
\newcommand{\bp}{\bolds{p}}
\newcommand{\bP}{\bolds{P}}
\newcommand{\bQ}{\bolds{Q}}
\newcommand{\bq}{\bolds{q}}
\newcommand{\br}{\bolds{r}}
\newcommand{\bR}{\bolds{R}}
\newcommand{\bs}{\bolds{s}}
\newcommand{\bS}{\bolds{S}}
\newcommand{\bt}{\bolds{t}}
\newcommand{\bT}{\bolds{T}}
\newcommand{\bu}{\bolds{u}}
\newcommand{\bU}{\bolds{U}}
\newcommand{\bv}{\bolds{v}}
\newcommand{\bV}{\bolds{V}}
\newcommand{\bw}{\bolds{w}}
\newcommand{\bW}{\bolds{W}}
\newcommand{\bx}{\bolds{x}}
\newcommand{\bX}{\bolds{X}}
\newcommand{\by}{\bolds{y}}
\newcommand{\bY}{\bolds{Y}}
\newcommand{\bz}{\bolds{z}}
\newcommand{\bZ}{\bolds{Z}}

\newcommand{\bzero}{\mathbf{0}}

% greek
\newcommand{\balpha}{\bolds{\alpha}}
\newcommand{\bbeta}{\bolds{\beta}}
\newcommand{\bdelta}{\bolds{\delta}}

\newcommand{\btau}{\bolds{\tau}}
\newcommand{\beps}{\bolds{\varepsilon}}
\newcommand{\btheta}{\bolds{\theta}}
\newcommand{\bomega}{\bolds{\omega}}
\newcommand{\brho}{\bolds{\rho}}
\newcommand{\bphi}{\bolds{\phi}}
\newcommand{\bpsi}{\bolds{\psi}}

\newcommand{\bSigma}{\bolds{\Sigma}}
\newcommand{\bsigma}{\bolds{\sigma}}
\newcommand{\bgamma}{\bolds{\gamma}}
\newcommand{\bGamma}{\bolds{\Gamma}}
\newcommand{\bLambda}{\bolds{\Lambda}}
\newcommand{\blambda}{\bolds{\lambda}}
\newcommand{\bDelta}{\bolds{\Delta}}
\newcommand{\bPsi}{\bolds{\Psi}}
\newcommand{\bmu}{\bolds{\mu}}
\newcommand{\bnu}{\bolds{\nu}}
\newcommand{\bxi}{\bolds{\xi}}
\newcommand{\boldeta}{\bolds{\eta}}
\newcommand{\bvarphi}{\bolds{\varphi}}


\newcommand{\joel}[1]{\textcolor{blue}{\textsf{[#1]}}}

% Track changes toggle: set to 1 to show tracked changes, 0 to hide them
\newcommand{\trackchanges}{1}

% Revised text environment - colors additions blue and indents when tracking is on
\newenvironment{revised}{%
  \if1\trackchanges%
    \color{blue}%
  \fi%
  \begin{quote}%
}{%
  \end{quote}%
}

\usepackage{xr}  % for cross-referencing if supplement and main manuscript are separate
\externaldocument{main}
\externaldocument{supplement}


\usepackage{graphicx} % Required for inserting images

\title{SHADE Responses}
\author{Joel Eliason}
\date{October 2025}

\begin{document}

\maketitle

We thank the reviewers for their thoughtful comments, which we believe have greatly improved this manuscript. Your feedback helped us clarify key methodological details and expand our comparisons with existing approaches. Below we have provided point-by-point responses to each comment. For easy reference, all manuscript changes are highlighted in blue, in both the responses here as well as the revised manuscript. We greatly appreciate the time and care you brought to reviewing this work!

\section{Reviewer Responses}

\subsection{Reviewer 1}
The authors present a Bayesian probabilistic model of the spatial relationship between cell types identified by multiplexed immunofluorescence, and presumably other assays such as spatial transcriptomics. The modeling choices appear sensible, with the spatial distribution of cells of a particular type described by an inhomogeneous Poisson point process whose intensity values are described as a log-linear function of the density of neighboring cells of other types. This method is limited to modeling discrete cell types or states, though it's a flexible enough model that one could imagine extending this to model dependence on marker intensities, which the authors suggest as a future extension. The use of a hierarchical model is an especially nice feature, as analysis of multiple replicates seems to be an afterthought too often in spatial experiments.

With that said, there are a few significant areas I hope the authors will give more consideration to in a future revision:

\begin{enumerate}
\item For inference, the method avoids computing the integral term of the Poisson point process likelihood and substitutes an approximate scheme based on logistic regression classification of observed cells versus homogenous Poisson noise. This is motivated at least in part by computational efficiency concerns, yet there doesn't appear to be any discussion of computational efficiency. Spatial technologies are scaling now to hundreds of thousands of cells per sample, so a natural question the authors ought to address is how this method scales. For example, Equation 3 scales quadratically with the number of cells if implemented naively, which would quickly become problematic for many thousands of cells.

\textbf{Response}: We added a discussion of computational scalability to Section~\ref{sec:model_estimation}, including empirical timing experiments demonstrating tractability for large-scale datasets:

\begin{revised}
Feature construction for each focal cell type $B$ involves evaluating inter-cell distances between observed and dummy focal locations and all non-focal cells. This step scales as $\mathcal{O}(n_{\mathrm{focal}} \times n_{\mathrm{source}})$, which is quadratic in the total number of cells when the focal and source sets are of similar size. However, this operation is implemented using the optimized \texttt{crossdist} routine from \texttt{spatstat.geom}, which efficiently computes pairwise distances in compiled code. The resulting distance matrix is reused across all basis functions $\{\phi_p\}$ and source types $\{A_k\}$, so the dominant cost occurs only once per focal type.

To assess the practical runtime implications, we performed a timing experiment varying the total number of cells from 5,000 to 250,000 using variational inference (Supplement, Section~\ref{sec:timing_experiments}). Feature construction time scaled as $O(n^{1.46})$ (empirical exponent from log-log regression), while total model fitting time scaled as $O(n^{0.85})$ due to efficient distance matrix reuse. At 100,000 cells, total fitting time was approximately 36 seconds; at 250,000 cells, approximately 133 seconds. These benchmarks demonstrate that SHADE remains computationally tractable for large-scale multiplexed imaging studies.

\end{revised}

\item What also frustrates interpretation is that every pair of cells has a negative association at very short distances due to spatial crowding (since cells are not really points, there's a limit to how close their centroids can be). Since most of the spatial associations detected appear to be pretty close range ones, it makes negative associations in particular hard to determine. It's not obvious at what distance we should interpret negative numbers as due to crowding versus due to cellular organization. It wasn't clear to me why there couldn't be a term in the regression to capture this generic crowding effect. Maybe more ambitiously, if full segmentation boundaries are available, cellular distances could be computed as boundary-to-boundary avoiding this issue.

\textbf{Response}: We agree that apparent negative associations at very short distances reflect geometric centroid exclusion rather than biologically meaningful repulsion. We now predefine a minimum interaction radius to avoid attributing geometric crowding to biological interactions, as detailed in Section~\ref{sec:sic}:

\begin{revised}
Because cell centroids cannot occur arbitrarily close in space, very short distances (below approximately 1-2 cell diameters) can reflect geometric crowding, cell-cell contact, or segmentation artifacts, making biological interpretation ambiguous. We therefore predefine a minimum interaction radius $r_{\min} = 25~\mu$m (approximately 1.5-2 typical cell diameters) and report spatial interaction curves only for $r\ge r_{\min}$. The model is fit using all observed cell locations, but posterior summaries and spatial interaction curves are evaluated and reported only at distances $r \ge r_{\min}$ to focus interpretation on unambiguous intercellular spacing. All band-level posterior probabilities are computed on intervals $I\subseteq [r_{\min},\infty)$.

\end{revised}

As noted in the revised Discussion, future extensions could further reduce crowding effects using boundary-to-boundary distances:

\begin{revised}
Future extensions could incorporate functional covariates (marker intensity, proliferation, exhaustion scores) into the SIC framework, enabling joint analysis of spatial structure and functional state, or leverage cell boundary information rather than centroids to better capture contact-based interactions.

\end{revised}

\item Some of the positive interaction scores (e.g in Fig. 6 and 8) seem quite small. Since inference is done via sampling, their credible intervals should be available which might tell us what we should consider significant. Fig. 3 does appear to show some type of uncertainty region, though I don't see a description of what exactly it is. More broadly, I'd like to see some discussion of assessing the overall statistical significance of associations. I would imagine a common use case of SHADE is to test all pairs of annotated cell types in both directions. Should one then inspect plots of all $n^2$ pairs in both directions, which could be hundreds of pairs, or is there some way to compute an aggregate posterior probability of non-zero effects at some distance?

\textbf{Response}: We thank the reviewer for this important point about uncertainty quantification and screening strategies. We have made three key additions to the manuscript:

\textbf{Interpreting small effects:} We now explain in Section~\ref{sec:sic} how SIC values on the log scale correspond to multiplicative changes in cell density:

\begin{revised}
This curve represents the expected contribution of type \( A_k \) cells to the log-intensity of type \( B \) cells as a function of distance \( s \) from a type \( A_k \) cell. Here, log-intensity refers to the logarithm of the conditional intensity function $\lambda(v)$ (Equation~\ref{eq:cond_int_simplified}), which describes the expected spatial density of type $B$ cells (in cells per unit area) at any location $v$. Because the model is log-linear, SIC values quantify additive changes on the log-intensity scale, which correspond to multiplicative changes in actual cell density. For example, $\text{SIC} = 0.2$ at distance $s$ implies an $e^{0.2} \approx 1.22\times$ (or 22\%) increase in the expected local density of type $B$ cells at radius $s$ from a type $A_k$ cell, while $\text{SIC} = -0.3$ corresponds to an $e^{-0.3} \approx 0.74\times$ decrease (26\% reduction).

\end{revised}

\textbf{Simultaneous credible bands and screening strategies:} We added a new subsection (Section~\ref{sec:uncertainty}) describing uncertainty quantification and prioritization approaches for exploratory analyses with many cell type pairs:

\begin{revised}
To quantify uncertainty in estimated SICs and assess statistical significance, we employ simultaneous 95\% credible bands that account for multiple comparisons across the distance domain. Unlike pointwise credible intervals, simultaneous bands provide joint coverage across all distances within a specified range, offering stronger protection against false discoveries. Statistical significance can be assessed by examining whether the simultaneous band excludes zero over a distance range of interest. Full implementation details are provided in the Supplement (\secref{ssec:sim_band}).

In exploratory analyses involving many cell type pairs ($K(K-1)$ directed pairs for $K$ cell types), we propose summary measures to facilitate prioritization: peak location and magnitude (identifying where the strongest interaction occurs), persistence over biologically relevant distance ranges (quantifying consistent associations within pre-specified intervals), and overall strength (integrating absolute effect sizes over significant regions). These measures can be computed across all source--target pairs and visualized as heatmaps for systematic comparison. Formal definitions are provided in the Supplement (\secref{ssec:screening_supplement}).

\end{revised}

All SIC figures now display simultaneous credible bands. Statistical significance is assessed by checking whether bands exclude zero over a distance range of interest.


\end{enumerate}

Minor issues:

\begin{enumerate}
\item Some acronyms are used without being defined (e.g. CTLs, TAMs). These are common immunological terms but still should be defined to avoid ambiguity.

\textbf{Response}: We have now defined all acronyms upon first use in the manuscript. CTLs (cytotoxic T lymphocytes, CD8\textsuperscript{+} T cells) and TAMs (tumor-associated macrophages, CD163\textsuperscript{+}) are now explicitly defined in the Results section.

\item Figure 10 caption needs more explanation. I'm guessing the black dots are observed cells of the labeled type, but it's not clear.

\textbf{Response}: We have clarified the caption for Figure~\ref{fig:example_pred}:

\begin{revised}
Heat maps show predicted log-intensity; black dots indicate observed cell locations for each target cell type.

\end{revised}
\end{enumerate}

\subsection{Reviewer 2}
The manuscript by Eliason et al. describes a method and software tool for quantitative analysis of point patterns describing cell types derived from multiplexed imaging data. The main output of the tool is the "spatial interaction curve", a distance-dependent function of the (directional) intensity of the influence of one cell-type on another regarding spatial positioning, e.g. co-localization of immune cells with tumor cells. The method is demonstrated using both synthetic data and application to a suitable published data set.

The described method appears to be a valuable contribution to a research field that indeed is short on methods with quantitative output suitable for systematic evaluation of large data sets (e.g. patient cohorts). The paper is well-written and the applications are well-chosen. Nevertheless, a number of points are not quite clear to me, especially regarding interpretation and statistical analysis of SIC curves, and I am a bit concerned on whether the analysis of synthetic data really addresses the most relevant points - please see details below with reference to text fragments:

\begin{enumerate}
\item Introduction: "... yet existing spatial models typically analyze images independently, potentially overlooking heterogeneity across patients", and similarly abstract: "Many analysis methods ... treat images independently".

Generally, I feel such statements should be made explicit and substantiated with references. Also, I do not quite understand the point, why would analyzing images independently automatically mean that patient heterogeneity is not studied? And is Shade not actually based on computing statistics (SIC curves) on individual images? In the current form of the ms, I do not find any follow-up on this point in the results or discussion sections, please clarify.

\textbf{Response}: We agree with the reviewer that our original statement was overstated. The Introduction now clarifies the distinction: independent analyses can descriptively characterize heterogeneity by comparing per-image summaries, but cannot formally decompose variance components or perform partial pooling within a unified framework. The revised Introduction now provides explicit references and explains this nuance:

\begin{revised}
standard spatial summary statistics such as Ripley's $K$-function and the $G$-cross function compute estimates separately for each image without explicit hierarchical structure~\citep{baddeley_spatial_2015,diggle_statistical_2013}, limiting their ability to formally model and partition heterogeneity across biological scales. While such methods can describe variability by computing per-image statistics, they do not enable joint estimation with partial pooling or explicit decomposition of variance components across images, patients, and cohorts

\end{revised}

The reviewer correctly notes that SHADE does compute image-level SICs. Results Section~\ref{sec:how_vary} now clarifies that these are estimated within a joint hierarchical framework:

\begin{revised}
Unlike descriptive spatial statistics that compute separate summaries for each image, SHADE's hierarchical Bayesian model jointly estimates SICs at the image, patient, and cohort levels through partial pooling. This enables formal decomposition and quantification of variability at each biological scale, allowing us to assess both intra- and inter-patient variability and distinguish conserved from patient-specific spatial patterns.

\end{revised}

Simulation evidence in Section~\ref{sec:sim_comparison} now demonstrates the benefits of this hierarchical approach:

\begin{quote}
We found that on average, and across all spatial scales, the average RMSE of \( \delta_{t_1 \to t_2}^{(m,p)} \) coefficients was lower for those estimated by the full hierarchical model than the model that fit images individually, most significantly so for closer-range coefficients (\tabref{stab:hier_rmse}).

\end{quote}

\begin{revised}
When source density is high (regardless of target density), SHADE achieves excellent median power (100\%) across all conditions, substantially outperforming envelope tests when target density is low (SHADE 100\% vs. G-cross 77\%, K-cross 73\%). Abundant source cells provide strong conditioning information that hierarchical pooling can effectively leverage. Conversely, when source density is low, performance depends on whether multiple images are available: with 2--3 images per patient, median power reaches 100\%; with only 1 image, power drops to 31\% as limited conditioning information per image prevents effective pooling.

\end{revised}

The Discussion now further contextualizes SHADE's hierarchical approach relative to traditional spatial statistics:

\begin{revised}
A key advantage of SHADE over traditional spatial summary statistics is its ability to formally model heterogeneity across biological scales through hierarchical partial pooling, enabling variance decomposition and more nuanced biological inference than post-hoc image-by-image comparisons.

\end{revised}

\item Methods (p. 8), "The SIC summarizes ... This curve represents the expected contribution of type $A_k$ cells to the log-intensity...".

The term "log-intensity" appears for the first time at this point, and lacks a direct interpretation. What does log-intensity of a cell type physically mean, is it related to the spatial cell density? Or would it be possible to normalize it to the average cell-density in the image, or a set of images? This is critical for interpretation of the SIC curves where log-intensity appears on the y-axis throughout the paper.

\textbf{Response}: We have added a detailed explanation of log-intensity and normalization in Section~\ref{sec:sic}.

\begin{revised}
This curve represents the expected contribution of type \( A_k \) cells to the log-intensity of type \( B \) cells as a function of distance \( s \) from a type \( A_k \) cell. Here, log-intensity refers to the logarithm of the conditional intensity function $\lambda(v)$ (Equation~\ref{eq:cond_int_simplified}), which describes the expected spatial density of type $B$ cells (in cells per unit area) at any location $v$. Because the model is log-linear, SIC values quantify additive changes on the log-intensity scale, which correspond to multiplicative changes in actual cell density. For example, $\text{SIC} = 0.2$ at distance $s$ implies an $e^{0.2} \approx 1.22\times$ (or 22\%) increase in the expected local density of type $B$ cells at radius $s$ from a type $A_k$ cell, while $\text{SIC} = -0.3$ corresponds to an $e^{-0.3} \approx 0.74\times$ decrease (26\% reduction).

\end{revised}

\begin{revised}
The intercept term ($\beta_0$) captures baseline log-intensity, effectively normalizing to average target cell density. SIC values then quantify deviations from this baseline as a function of proximity to source cells, ensuring curves are directly comparable across images and patients despite differences in overall cell abundance.

\end{revised}

\item Methods (p. 9), "A strong A->B SIC indicates that the presence of A is statistically predictive ..." (and similar statements in the following).

It is not clear to me what statistically predictive means here. Is it possible to construct a statistical test for SIC curves different from spatial randomness, or for SIC curves differing between different image sets? The next section about "Multilevel Bayesian Model" suggests that the authors aim for statistical analysis, but I have trouble finding any clear statistical statements, also in the following chapters where synthetic data and image data sets are analyzed. Figures 4 and 5 show confidence intervals but statements on statistical significance are missing, I suggest evaluating significance using bootstrapping procedures. In Figure 4, it seems that all sown curves are within the confidence of the null (CSR) model, does that mean that the generated synthetic data cannot be distinguished from CSR?

\textbf{Response}: We have made statistical inference much more explicit throughout the manuscript.

\textbf{Bayesian inference and uncertainty quantification:} We added Section~\ref{sec:uncertainty} describing simultaneous 95\% credible bands that control family-wise error across distances:

\begin{revised}
To quantify uncertainty in estimated SICs and assess statistical significance, we employ simultaneous 95\% credible bands that account for multiple comparisons across the distance domain. Unlike pointwise credible intervals, simultaneous bands provide joint coverage across all distances within a specified range, offering stronger protection against false discoveries. Statistical significance can be assessed by examining whether the simultaneous band excludes zero over a distance range of interest. Full implementation details are provided in the Supplement (\secref{ssec:sim_band}).

In exploratory analyses involving many cell type pairs ($K(K-1)$ directed pairs for $K$ cell types), we propose summary measures to facilitate prioritization: peak location and magnitude (identifying where the strongest interaction occurs), persistence over biologically relevant distance ranges (quantifying consistent associations within pre-specified intervals), and overall strength (integrating absolute effect sizes over significant regions). These measures can be computed across all source--target pairs and visualized as heatmaps for systematic comparison. Formal definitions are provided in the Supplement (\secref{ssec:screening_supplement}).

\end{revised}

\textbf{Clarifying ``statistically predictive'':} We updated Section~\ref{sec:predictive} explaining what predictive associations mean:

\begin{revised}
Our modeling framework captures \emph{predictive} spatial associations, not biological causation. A strong \( A \to B \) SIC indicates that type $A$ cell locations are statistically predictive of local type $B$ density, formally meaning that the conditional intensity $\lambda(v \mid X_A)$ differs from the marginal intensity $\lambda_B(v)$. While SICs quantify the strength and direction of spatial predictability, they do not establish causal biological effects. The model can incorporate spatial covariates (e.g., distance to tumor margin) to adjust for confounding from first-order intensity effects.

\end{revised}

\textbf{Testing and comparison:} As mentioned previously, to test departure from spatial randomness, we now check if simultaneous bands exclude zero. To compare patient groups, we compute difference curves with simultaneous bands e.g. (Figure~\ref{fig:sic_CRC_diff}); if bands exclude zero over a distance range, groups differ significantly. All SIC figures now display these bands, and Results sections explicitly state when bands exclude zero.

\textbf{Figure 4 and CSR:} Figure~\ref{fig:method_comparison} shows one simulation example: while $G$-cross and $L$-cross envelopes sometimes contain the observed curves, SHADE's credible bands correctly exclude zero where the ground truth exhibits non-zero interactions, demonstrating that the data can be distinguished from CSR. Panel (d) also shows that SHADE Hierarchical (red) estimates the ground truth more accurately than SHADE Flat (blue), illustrating the benefit of hierarchical pooling. Across all replicates, SHADE achieves high detection rates (Figure~\ref{fig:gx_detections}).

\item In the same context, it would make sense to include more methods for comparison (in addition to G-cross), for example Ripley's K-function is mentioned in the introduction.

\textbf{Response}: We have now added Ripley's cross-type $K$-function ($K_{AB}$, specifically the $L$-cross transformation) to both the simulation comparison and colorectal cancer analysis.

In the simulation study (Section~\ref{sec:sim_comparison}), we added a comparison of SHADE against K-cross envelope tests:

\begin{revised}
For comparison, we evaluated three baseline approaches: (1) $G$-cross envelope tests~\citep{baddeley_spatial_2015}, testing whether the observed nearest-neighbor distribution falls outside 95\% global envelopes constructed from 99 completely spatially random (CSR) simulations; (2) $K$-cross (specifically, $L$-cross) envelope tests, using cumulative counts rather than nearest-neighbor distances; and (3) a `Flat' model that estimates SICs independently for each image without hierarchical pooling. Type I error rates were assessed using null simulations with zero spatial interactions.

\end{revised}

Results (Figure~\ref{fig:gx_detections}) now show SHADE outperforms both $G$-cross and $K$-cross in detecting associations under low cell densities and limited sampling.

As now shown in Section~\ref{sec:gx_comparison}:

\begin{revised}
The $G$-cross-based analysis (Figure~\ref{fig:gx_comparison}) reveals several patterns of cell--cell clustering between CLR and DII patients, but notable differences emerge when we compare these results with SHADE-derived SICs. For instance, TAM--granulocyte interactions showed strong, statistically significant differences at all examined distances (20--60 microns), with higher clustering in DII tumors. This suggests a prominent granulocyte-TAM association in DII tumors according to $G$-cross. However, SHADE did not highlight this pair as a major differential interaction between groups, indicating that the clustering of granulocytes around TAMs may not translate into strong directional interactions when adjusting for other cell types or when modeled hierarchically. Conversely, SHADE identified vasculature as a key source driving differences in clustering of both CTLs and memory CD4+ T cells between CLR and DII groups, reflecting enhanced immune surveillance or trafficking around blood vessels in CLR tumors. This vasculature effect is less evident in the $G$-cross analysis, where vasculature-related interactions showed limited significance and mostly small effect sizes.

\end{revised}

We also added Section~\ref{sec:mfpca_comparison} comparing SHADE with mFPCA analysis of $G$-cross and $L$-cross functions:

\begin{revised}
To provide additional context for SHADE's results and explore complementary perspectives on group-level spatial organization, we compared SHADE's group-level SICs with functional data analysis (FDA) of marginal pairwise summary statistics. Specifically, we applied multilevel functional principal component analysis (mFPCA)~\citep{wrobelMxfdaComprehensiveToolkit2024} separately to CLR and DII patient groups, analyzing $G$-cross and $L$-cross functions computed for each tissue section.

Unlike SHADE, which models conditional intensities adjusting for all cell types simultaneously within a generative point process framework, mFPCA decomposes functional variation in observed summary statistics across hierarchical levels (images nested within patients). This approach does not provide a probabilistic model of the underlying point process, but instead characterizes dominant modes of variation in the empirical $G$-cross and $L$-cross curves. The resulting group-level mean functions with variability bands (mean $\pm$ 1 SD from the first functional principal component) provide a complementary view of pairwise spatial associations. Note that these bands represent descriptive variation captured by the first PC and do not have coverage properties analogous to confidence or credible intervals.

\end{revised}

As now shown in Section~\ref{sec:mfpca_comparison}:

\begin{revised}
For instance, SHADE identified greater CTL clustering around tumor cells in DII patients across all distances (Figure~\ref{fig:sic_crc_all}), while $G$-cross mFPCA curves show the opposite pattern, with CLR patients exhibiting stronger CTL-tumor associations. This discrepancy likely reflects SHADE's multivariate adjustment: after controlling for other cell types, the conditional CTL-tumor interaction is stronger in DII, whereas marginal pairwise statistics capture different aspects of spatial organization. Similarly, SHADE detected greater granulocyte-TAM clustering in DII patients, but $G$-cross mFPCA shows minimal group differences for this interaction, suggesting that SHADE's hierarchical modeling and conditional framework reveals patterns that are subtle or absent in marginal analyses. Additionally, SHADE identified stronger CTL repulsion from CAFs at medium-long range in CLR patients, a pattern not evident in marginal $G$-cross analyses. These discrepancies highlight how multivariate adjustment and hierarchical pooling can reveal conditional dependencies that differ from marginal pairwise associations.

\end{revised}

\begin{revised}
This comparison illustrates the complementary nature of SHADE and FDA approaches: SHADE's conditional modeling isolates directional effects while adjusting for confounders, whereas mFPCA characterizes marginal pairwise variation without multivariate adjustment. Both perspectives contribute to understanding group-level spatial organization in the tumor microenvironment.

\end{revised}

\item Simulation Studies (p. 12), "Spatial patterns were simulated ... with source points generated from a homogeneous Poisson process and target points influenced by spatial interaction curves ..."

This sounds to me as if the synthetic data were generated under exactly the same assumptions used for deriving the SIC curves proposed for analysis, is that correct? Would that not by construction give an advantage to the method proposed by the authors when comparing different analysis methods? It would be important to test the method also using other sources of synthetic data, e.g. starting from Bridson sampling or using a Gauss filter after the Poisson process.

\textbf{Response}: The reviewer is correct that our primary simulation generates data under the same conditional Poisson framework that SHADE assumes. This is standard for initial method evaluation—it allows us to isolate the effects of hierarchical structure, cell density, and sampling variation on estimation accuracy. Even under these ideal conditions, Section~\ref{sec:sim_hier} demonstrates that hierarchical pooling substantially improves inference:

\begin{quote}
We found that on average, and across all spatial scales, the average RMSE of \( \delta_{t_1 \to t_2}^{(m,p)} \) coefficients was lower for those estimated by the full hierarchical model than the model that fit images individually, most significantly so for closer-range coefficients (\tabref{stab:hier_rmse}).

\end{quote}

Section~\ref{sec:sim_comparison} now demonstrates that SHADE's performance relative to envelope tests depends on source cell density and the availability of multiple images for hierarchical pooling:

\begin{revised}
When source density is high (regardless of target density), SHADE achieves excellent median power (100\%) across all conditions, substantially outperforming envelope tests when target density is low (SHADE 100\% vs. G-cross 77\%, K-cross 73\%). Abundant source cells provide strong conditioning information that hierarchical pooling can effectively leverage. Conversely, when source density is low, performance depends on whether multiple images are available: with 2--3 images per patient, median power reaches 100\%; with only 1 image, power drops to 31\% as limited conditioning information per image prevents effective pooling.

\end{revised}

\textbf{Robustness to spatial confounding:} To address model misspecification, we added simulations testing SHADE's performance with unmeasured spatial heterogeneity (Section~\ref{sec:sim_comparison}):

\begin{revised}
We also tested SHADE's performance when the model is misspecified due to unmeasured spatial heterogeneity—specifically, discrete tissue compartments (e.g., tumor islands, stromal regions) that create baseline density differences independent of source-target interactions (Figure~\ref{fig:compartment_robustness}; \secref{ssec:compartment_confounder}). Results reveal regime-dependent bias: when both cell types are abundant, SHADE achieves perfect detection power but exhibits elevated Type I error rates (11.7--17.1\%) and severely undercovers (43--52\% vs.\ expected 95\%), incorrectly attributing compartment effects to source-target interactions. When target density is low, wider credible bands provide partial robustness (82--93\% coverage, 1.7--5.8\% Type I error). These findings indicate that unmeasured spatial structure can produce substantial bias in high-density scenarios, suggesting the need for explicit compartment modeling or sensitivity analyses when such heterogeneity is suspected.

\end{revised}

\end{enumerate}

Minor points:

\begin{enumerate}
\item Introduction, "While our approach is not a direct extension of the multitype Gibbs points model ..., it is certainly inspired by it" (and similar statements in Methods): It would be very helpful to formally introduce these methods the paper is based on, possibly in a supplement, and then show direct comparisons on a suitable test problem. At the moment, it is not straight-forward to evaluate the extensions and modifications introduced by the authors.

\textbf{Response}: We have added Supplement Section~\ref{sec:gibbs_background} ("Background: Multitype Gibbs Point Process Models and Relationship to SHADE") providing: (1) formal introduction to multitype Gibbs point process (MGPP) models and their symmetry constraint that prevents directional association modeling, (2) description of hierarchical Gibbs models with type ordering~\citep{hougmander_multitype_1999} that enable asymmetry via ordered dependencies, (3) SHADE's three key extensions (asymmetric SICs, flexible basis functions, multilevel Bayesian structure), and (4) comparison table (Table~\ref{tab:mgpp_vs_shade}) contrasting MGPPs, hierarchical Gibbs models, and SHADE. The Introduction now references this supplemental material.

\item The introduction is quite long and contains sections that might be better placed in the discussion, such as specific comparison to other methods.
\joel{Yes}
\item Methods (p. 7), "... assuming that ... follows an inhomogeneous Poisson point process": Could the authors indicate what that assumption implies in the context of typical applications of the new method (e.g., a Poisson process typically is based on a rare-events argument)? And would there be reasonable alternative choices for the probability distribution?

\textbf{Response}: We have expanded Section~\ref{sec:model_setup} to clarify what the inhomogeneous Poisson point process assumption allows and why it is the appropriate modeling choice:

\begin{revised}
This allows the expected density of target cells to vary flexibly across space as a function of source cell locations and covariates, enabling the framework to represent complex spatial patterns including clustering, repulsion, and distance-dependent associations. While alternative models exist (e.g., Cox processes, cluster processes, Gibbs processes), the inhomogeneous Poisson process provides the best balance of flexibility, interpretability, and computational feasibility for our application~\citep{baddeley_spatial_2015}

\end{revised}

\end{enumerate}

\subsection{Reviewer 3}

This manuscript presents a method for can detect spatial interactions (co-localizations) that are directional (asymmetrical). The method allows for modeling jointly multiple images simultaneously from multiple subjects and cohorts within a Bayesian hierarchical model. It also allows for the assessment of these interactions at a range a distances using a curve (spatial interaction curve, SIC). The key innovation of this work is the asymmetric aspect and the ability to measure interactions as various distances and not one pre-specified distance. They compare the method to G cross and apply to a large colon cancer study published in Cell in 2020. The manuscript was well written. Below are my comments/questions about the approach and presentation of results.

\begin{enumerate}
\item How does the SIC compare to using functional data analysis (FDA) of the spatial curves that ones gets from K or G? For example, using basis functions that look similar in formulation to functional data analysis and functional principal component analysis?

\textbf{Response}: While both SHADE and FDA methods represent spatial relationships as smooth curves using basis function expansions, they differ fundamentally in what they model and how they leverage hierarchical structure. As now detailed in Supplement Section~\ref{sec:mfpca_methods}:

\begin{revised}
SHADE directly models the underlying point process through the conditional intensity $\lambda(v \mid X_{A_1}, \ldots, X_{A_K})$ (Equation~\ref{eq:cond_int_simplified}), leveraging the point-level likelihood for inference. In contrast, functional data analysis (FDA) approaches applied to spatial statistics (e.g., mFPCA of $G$-cross or $K$-cross functions) first compute per-image summary statistics, then analyze the resulting curves in a second stage—a two-stage procedure that does not propagate uncertainty from point pattern estimation into functional inference. While both methods can leverage hierarchical structure, SHADE's generative modeling framework directly integrates the spatial point process likelihood, enabling simultaneous estimation of spatial interactions and variance components across biological scales through partial pooling.
\end{revised}

As now discussed in Section~\ref{sec:mfpca_comparison}:

\begin{revised}
To provide additional context for SHADE's results and explore complementary perspectives on group-level spatial organization, we compared SHADE's group-level SICs with functional data analysis (FDA) of marginal pairwise summary statistics. Specifically, we applied multilevel functional principal component analysis (mFPCA)~\citep{wrobelMxfdaComprehensiveToolkit2024} separately to CLR and DII patient groups, analyzing $G$-cross and $L$-cross functions computed for each tissue section.

Unlike SHADE, which models conditional intensities adjusting for all cell types simultaneously within a generative point process framework, mFPCA decomposes functional variation in observed summary statistics across hierarchical levels (images nested within patients). This approach does not provide a probabilistic model of the underlying point process, but instead characterizes dominant modes of variation in the empirical $G$-cross and $L$-cross curves. The resulting group-level mean functions with variability bands (mean $\pm$ 1 SD from the first functional principal component) provide a complementary view of pairwise spatial associations. Note that these bands represent descriptive variation captured by the first PC and do not have coverage properties analogous to confidence or credible intervals.

\end{revised}

And also now in Section~\ref{sec:mfpca_comparison}:

\begin{revised}
This comparison illustrates the complementary nature of SHADE and FDA approaches: SHADE's conditional modeling isolates directional effects while adjusting for confounders, whereas mFPCA characterizes marginal pairwise variation without multivariate adjustment. Both perspectives contribute to understanding group-level spatial organization in the tumor microenvironment.

\end{revised}

Lastly, as now noted in the Discussion:

\begin{revised}
A key advantage of SHADE over traditional spatial summary statistics is its ability to formally model heterogeneity across biological scales through hierarchical partial pooling, enabling variance decomposition and more nuanced biological inference than post-hoc image-by-image comparisons.

\end{revised}

\item It would be good to compare you approach to using FDA on the G or K statistics, as implemented in the R package mxfda(Wrobel et al. 2024). Additionally, should compare to not just G but also K as this has been used often in practice and has been reported to have better discrimination ability for detecting co-localization than G(Soupir et al. 2025).

\textbf{Response}: We have added both $G$-cross and Ripley's $L$-cross function (transformation of the $K$-function) to our analysis. As now described in Section~\ref{sec:mfpca_comparison}, we implemented a visual comparison with mxfda~\citep{wrobelMxfdaComprehensiveToolkit2024} (detailed methodology in Supplement Section~\ref{sec:mfpca_methods}):

\begin{revised}
To provide additional context for SHADE's results and explore complementary perspectives on group-level spatial organization, we compared SHADE's group-level SICs with functional data analysis (FDA) of marginal pairwise summary statistics. Specifically, we applied multilevel functional principal component analysis (mFPCA)~\citep{wrobelMxfdaComprehensiveToolkit2024} separately to CLR and DII patient groups, analyzing $G$-cross and $L$-cross functions computed for each tissue section.

Unlike SHADE, which models conditional intensities adjusting for all cell types simultaneously within a generative point process framework, mFPCA decomposes functional variation in observed summary statistics across hierarchical levels (images nested within patients). This approach does not provide a probabilistic model of the underlying point process, but instead characterizes dominant modes of variation in the empirical $G$-cross and $L$-cross curves. The resulting group-level mean functions with variability bands (mean $\pm$ 1 SD from the first functional principal component) provide a complementary view of pairwise spatial associations. Note that these bands represent descriptive variation captured by the first PC and do not have coverage properties analogous to confidence or credible intervals.

\end{revised}

\begin{revised}
We computed $G$-cross and $L$-cross functions for all 15 source-target pairs at distances from 0 to 75~$\mu$m, ran mFPCA separately for each group, and extracted the population mean curves with variability bands. Full mFPCA results for all pairs are provided in Figures~\ref{fig:mfpca_gcross_grid} and \ref{fig:mfpca_lcross_grid} for $G$-cross and $L$-cross, respectively. Visual comparison of SHADE SICs (Figure~\ref{fig:sic_crc_all}) with mFPCA group curves reveals both concordance and informative differences.

\end{revised}

\begin{revised}
For instance, SHADE identified greater CTL clustering around tumor cells in DII patients across all distances (Figure~\ref{fig:sic_crc_all}), while $G$-cross mFPCA curves show the opposite pattern, with CLR patients exhibiting stronger CTL-tumor associations. This discrepancy likely reflects SHADE's multivariate adjustment: after controlling for other cell types, the conditional CTL-tumor interaction is stronger in DII, whereas marginal pairwise statistics capture different aspects of spatial organization. Similarly, SHADE detected greater granulocyte-TAM clustering in DII patients, but $G$-cross mFPCA shows minimal group differences for this interaction, suggesting that SHADE's hierarchical modeling and conditional framework reveals patterns that are subtle or absent in marginal analyses. Additionally, SHADE identified stronger CTL repulsion from CAFs at medium-long range in CLR patients, a pattern not evident in marginal $G$-cross analyses. These discrepancies highlight how multivariate adjustment and hierarchical pooling can reveal conditional dependencies that differ from marginal pairwise associations.

\end{revised}

\begin{revised}
This comparison illustrates the complementary nature of SHADE and FDA approaches: SHADE's conditional modeling isolates directional effects while adjusting for confounders, whereas mFPCA characterizes marginal pairwise variation without multivariate adjustment. Both perspectives contribute to understanding group-level spatial organization in the tumor microenvironment.

\end{revised}

\item In the set-up of the model and approach in section 2.1, assume a inhomogeneous Poisson process. However, in the computational approximation using logistic regression you have a homogeneous Poisson process. In the logistic regression model fitting, how do you account for inhomogeneity? That is, is the analysis done within the tumor and stroma compartments of the tissue as different cells could have different intensity within these two tissue domains. If this is not explicitly accounted for in the model, I would recommend that analysis be done by tissue domains to limit the issue of inhomogeneity.

\textbf{Response}: As now described in Section~\ref{sec:model_setup}, the inhomogeneous Poisson point process assumption allows:

\begin{revised}
This allows the expected density of target cells to vary flexibly across space as a function of source cell locations and covariates, enabling the framework to represent complex spatial patterns including clustering, repulsion, and distance-dependent associations. While alternative models exist (e.g., Cox processes, cluster processes, Gibbs processes), the inhomogeneous Poisson process provides the best balance of flexibility, interpretability, and computational feasibility for our application~\citep{baddeley_spatial_2015}

\end{revised}

Regarding the logistic regression approximation (now in Section~\ref{sec:logistic_approx}):

\begin{revised}
Importantly, this approximation does not require the target cell process to be homogeneous—only the dummy/quadrature points are placed homogeneously for computational convenience. Target cells are still modeled as an inhomogeneous Poisson process via the conditional intensity $\lambda(v)$, with spatial variation captured through distance-based SIC features $\bq_{A_k}(v)$ and optional spatial covariates $\bz(v)$ (Equation~\ref{eq:cond_int_simplified}). While our later analysis does not explicitly model compartmental variation in baseline rates, the SIC features already capture local variation driven by proximity to different cell types. The framework is extensible to include compartment indicators or other spatially-varying features as additional covariates if discrete tissue domains are suspected to drive baseline intensity differences independent of cell-cell interactions.

\end{revised}

\item In the simulation study, please state how many simulated datasets were generated for each scenario. I think you have 30 simulations per scenario but not sure. If this is the case that used only 30 simulations per scenario, why so few of simulations? Possible to have 100 per scenario to get more precise estimates of power and type I error rate to compare the 3 approaches? Also, state in the simulation set-up the range of cell abundances that assessed and why 39 simulated used to estimate envelopes for G.

\textbf{Response}: As now described in Section~\ref{sec:sim_comparison}:

\begin{revised}
We simulated hierarchical point patterns (40 patients, 1--3 images per patient) with positive spatial interactions at multiple distance scales. We varied source cell density (15 vs. 150 cells per image) and target cell density (15 vs. 150 cells per image) across 50 replicates per condition.

\end{revised}

Each replication consists of multiple images (1--3 per patient), and hypothesis testing is performed at the image level, resulting in well over 100 image-level tests per scenario for precise power and Type I error estimates.

For envelope test calibration (now in Supplement Section~\ref{sec:sim_details}):

\begin{revised}
Global envelopes were constructed using $99$ Monte Carlo simulations of point patterns generated according to complete spatial randomness (CSR). The rank parameters were calibrated in preliminary null scenario simulations to achieve proper Type I error control: \texttt{nrank = 10} for $G$-cross (yielding nominal significance level $\alpha = 10/100 = 0.10$) and \texttt{nrank = 8} for $L$-cross (yielding $\alpha = 8/100 = 0.08$). These nominal levels achieved empirical Type I error rates of 2.5--8.8\% for $G$-cross and 1.3--6.7\% for $L$-cross across different density regimes (\secref{ssec:coverage_type1}).

\end{revised}

\item Please have a "null" scenario and present the type I error rate. In Figure 5 only presenting the power which can't be interpreted without knowing the approaches control the type I error rate.

\textbf{Response}: We agree this is an important check for calibration. We evaluated Type I error rates using null scenario simulations where target cells are generated with no spatial dependence on source cells (all SIC coefficients set to zero). As now discussed in Section~\ref{sec:sim_comparison}:

\begin{revised}
Coverage and type I error performance (Supplement Section~\ref{sec:coverage_type1}) reveal adaptive calibration. When source density is high (favorable for detection), SHADE trades calibration for sensitivity (55\% coverage, 100\% power, $<$1\% type I error). When both densities are low (unfavorable for detection), SHADE becomes appropriately conservative (94\% coverage, 28\% power, 4\% type I error). This contrasts with SHADE Flat, which maintains poor coverage (73\%) regardless of scenario and shows severely inflated type I error rates (28\%). Median type I error rates for SHADE Hierarchical with 2--3 images are well-controlled (0.8--7.5\%), comparable to envelope tests (G-cross: 2.5--8.8\%; K-cross: 1.3--6.7\%), though with higher variability (IQR up to 0.12), indicating occasional liberal inference under challenging conditions.

\end{revised}

Full results with figures are now presented in Supplement Section~\ref{sec:coverage_type1}.

\item Please present computational time to fit the model. The colon cancer data (I believe) is based on a TMA. How would this method scale for computing on whole tissue slides with millions of cells?

\textbf{Response}: The reviewer is correct that the colorectal cancer dataset consists of tissue microarray (TMA) cores with hundreds to thousands of cells per image. To address computational scalability for larger datasets, we added timing experiments (Supplement Section~\ref{sec:timing_experiments}), now discussed in Section~\ref{sec:model_estimation}:

\begin{revised}
Feature construction for each focal cell type $B$ involves evaluating inter-cell distances between observed and dummy focal locations and all non-focal cells. This step scales as $\mathcal{O}(n_{\mathrm{focal}} \times n_{\mathrm{source}})$, which is quadratic in the total number of cells when the focal and source sets are of similar size. However, this operation is implemented using the optimized \texttt{crossdist} routine from \texttt{spatstat.geom}, which efficiently computes pairwise distances in compiled code. The resulting distance matrix is reused across all basis functions $\{\phi_p\}$ and source types $\{A_k\}$, so the dominant cost occurs only once per focal type.

To assess the practical runtime implications, we performed a timing experiment varying the total number of cells from 5,000 to 250,000 using variational inference (Supplement, Section~\ref{sec:timing_experiments}). Feature construction time scaled as $O(n^{1.46})$ (empirical exponent from log-log regression), while total model fitting time scaled as $O(n^{0.85})$ due to efficient distance matrix reuse. At 100,000 cells, total fitting time was approximately 36 seconds; at 250,000 cells, approximately 133 seconds. These benchmarks demonstrate that SHADE remains computationally tractable for large-scale multiplexed imaging studies.

\end{revised}

Extrapolating from this empirical $O(n^{0.85})$ scaling to whole slide images with 1--2 million cells, we estimate total fitting times of 5--15 minutes per image using variational inference on a single core, which remains practical for large-scale studies. For MCMC inference, runtimes would be longer but could be accelerated via within-chain parallelization or GPU acceleration.

\item In the hierarchical model (section 2.2), please present where in the model you assess for differences by a factor/exposure (i.e., CLR and DII as used in the colon study).

\textbf{Response}: As now described in Section~\ref{sec:multilevel_model}:

\begin{revised}
When comparing spatial organization across biological groups (e.g., treatment responders vs. non-responders, different tumor subtypes), each group is modeled as a separate cohort with its own cohort-level parameters $\psi_{A_k}^{(g,p)}$. Differences between groups are then assessed by comparing the posterior distributions of these cohort-level parameters, as quantified through SICs (Equation~\ref{eq:sic}) and their associated simultaneous credible bands.

\end{revised}

Specifically, for each source--target cell type pair, we compute the difference in cohort-level SICs between groups and examine whether the simultaneous credible bands around this difference exclude zero (e.g., Figure~\ref{fig:sic_CRC_diff}).

\item Please provide details on the analysis of the colon cancer data how you determined if the clustering was significant taking into account random chance? I don't see any information presented on how this would be done beyond just reference to figure that shows slightly different curves (Figure 6, Figure 9). In figure 9, looks like set arbitrary level at a score of 0.05.

\textbf{Response}: Under the inhomogeneous Poisson process model, the null hypothesis of "no spatial association" corresponds to SIC = 0 (i.e., the target cell intensity is conditionally independent of the source cell locations). Spatial clustering or repulsion is detected when the estimated SIC is significantly different from zero. As now described in Section~\ref{sec:uncertainty}:

\begin{revised}
To quantify uncertainty in estimated SICs and assess statistical significance, we employ simultaneous 95\% credible bands that account for multiple comparisons across the distance domain. Unlike pointwise credible intervals, simultaneous bands provide joint coverage across all distances within a specified range, offering stronger protection against false discoveries. Statistical significance can be assessed by examining whether the simultaneous band excludes zero over a distance range of interest. Full implementation details are provided in the Supplement (\secref{ssec:sim_band}).

In exploratory analyses involving many cell type pairs ($K(K-1)$ directed pairs for $K$ cell types), we propose summary measures to facilitate prioritization: peak location and magnitude (identifying where the strongest interaction occurs), persistence over biologically relevant distance ranges (quantifying consistent associations within pre-specified intervals), and overall strength (integrating absolute effect sizes over significant regions). These measures can be computed across all source--target pairs and visualized as heatmaps for systematic comparison. Formal definitions are provided in the Supplement (\secref{ssec:screening_supplement}).

\end{revised}

When the simultaneous credible band excludes zero over a distance range, we conclude that the spatial association is statistically significant at that range. All SIC figures display simultaneous 95\% credible bands computed using the method now detailed in Supplement Section~\ref{sec:sim_band}.

Note: The "0.05 level" mentioned in Figure~\ref{fig:sic_CRC_diff} refers to differences in log-intensity exceeding 0.05 between groups—this is a practical effect size threshold for highlighting substantive differences, not a statistical significance cutoff.

\item G assumes homogeneity in the point process. Please look at results for G by tissue compartment (tumor, stroma, etc) to somewhat control for difference in intensity of different cell types by tissue domain.

\textbf{Response}: The CRC dataset lacks compartment annotations, and the \texttt{mxfda} package only supports standard $G$-cross and not \texttt{Gcross.inhom}, the inhomogeneous version of $G$-cross from \texttt{spatstat}. Our CRC analysis does not incorporate spatial covariates, so both methods make homogeneity assumptions in this application. However, we have added an investigation of the impact of unmeasured compartment structure through simulation (Section~\ref{sec:sim_comparison}; Supplement Section~\ref{sec:compartment_confounder}):

\begin{revised}
We also tested SHADE's performance when the model is misspecified due to unmeasured spatial heterogeneity—specifically, discrete tissue compartments (e.g., tumor islands, stromal regions) that create baseline density differences independent of source-target interactions (Figure~\ref{fig:compartment_robustness}; \secref{ssec:compartment_confounder}). Results reveal regime-dependent bias: when both cell types are abundant, SHADE achieves perfect detection power but exhibits elevated Type I error rates (11.7--17.1\%) and severely undercovers (43--52\% vs.\ expected 95\%), incorrectly attributing compartment effects to source-target interactions. When target density is low, wider credible bands provide partial robustness (82--93\% coverage, 1.7--5.8\% Type I error). These findings indicate that unmeasured spatial structure can produce substantial bias in high-density scenarios, suggesting the need for explicit compartment modeling or sensitivity analyses when such heterogeneity is suspected.

\end{revised}

This analysis informs interpretation when compartment structure may be present, indicating that SHADE's performance depends on the density regime and suggesting caution when both cell types are abundant without explicit compartment modeling.
\item What does the MAD presented in Figure 7 look like between CLR and DII tumors?

\textbf{Response}: We have added an analysis comparing MAD measures of heterogeneity between CLR and DII patient groups, now discussed in Section~\ref{sec:how_vary}:

\begin{revised}
Comparing heterogeneity patterns between CLR and DII tumor subtypes reveals subtype-specific differences in variability (Supplement Figure~\ref{fig:mad_clr_dii}). At the patient level, DII tumors show greater between-patient heterogeneity than CLR tumors for several immune-related interactions, most notably memory CD4+ T cells with CAFs (MAD difference = 0.015), CTLs with tumor cells (0.014), and granulocytes with vasculature (0.014). Conversely, CLR tumors exhibit greater between-patient variability for granulocyte-CAF interactions (MAD difference = $-0.025$) and memory CD4+ T-TAM interactions ($-0.017$). At the image level, DII tumors show more within-patient heterogeneity for granulocyte-vasculature (MAD difference = 0.028) and memory CD4+ T-tumor interactions (0.017), while CLR tumors show greater within-patient variability for granulocyte interactions with hybrid E/M cells ($-0.034$) and CAFs ($-0.031$), and for CTL-CAF interactions ($-0.020$). These results demonstrate that tumor subtype affects not only the mean spatial interaction patterns (Section~\ref{sec:hot_cold}) but also their variability at both between-patient and within-patient scales. The particularly strong subtype-specific differences in granulocyte-related heterogeneity may reflect distinct modes of myeloid cell recruitment and spatial organization between the two tumor subtypes.

\end{revised}
\item A better literature review related to methods for co-localizations and those applied to mIF data in cancer should be in the introduction of the paper.

\textbf{Response}: We have expanded the Introduction to include a more comprehensive review of spatial co-localization methods applied to multiplexed imaging data in cancer:

\begin{revised}
standard spatial summary statistics such as Ripley's $K$-function and the $G$-cross function compute estimates separately for each image without explicit hierarchical structure~\citep{baddeley_spatial_2015,diggle_statistical_2013}, limiting their ability to formally model and partition heterogeneity across biological scales. While such methods can describe variability by computing per-image statistics, they do not enable joint estimation with partial pooling or explicit decomposition of variance components across images, patients, and cohorts

\end{revised}

Additionally, Section~\ref{sec:mfpca_comparison} includes a detailed visual comparison with mFPCA applied to $G$-cross and $L$-cross functions, demonstrating the complementary strengths of SHADE and FDA methods for analyzing multiplexed imaging data.

\end{enumerate}

Minor:
\begin{enumerate}
    \item Figure 5: I can’t seem to see the 20, 40, and 60 nm data. Also, add to caption what is hight and low levels.
\end{enumerate}


Soupir, A. C., I. V. Gadiyar, B. R. Helm, C. R. Harris, S. N. Vandekar, L. C. Peres, R. J. Coffey, J. Wrobel, S. Ma, and B. L. Fridley. 2025. 'Benchmarking Spatial Co-Localization Methods for Single-Cell Multiplex Imaging Data with Applications to High-Grade Serous Ovarian and Triple Negative Breast Cancer', Stat Data Sci Imaging, 2.
Wrobel, J., A. C. Soupir, M. T. Hayes, L. C. Peres, T. Vu, A. Leroux, and B. L. Fridley. 2024. 'mxfda: a comprehensive toolkit for functional data analysis of single-cell spatial data', Bioinform Adv, 4: vbae155.

\bibliographystyle{abbrvnat}
\bibliography{references}

\end{document}
