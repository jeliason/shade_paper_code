A key advantage of SHADE over traditional spatial summary statistics is its ability to formally model and quantify heterogeneity across biological scales. While methods such as Ripley's $K$-function and the $G$-cross function can compute separate estimates for each image and then compare them post-hoc, SHADE's hierarchical Bayesian framework enables joint estimation with partial pooling, explicit variance decomposition, and formal quantification of between-patient and within-patient variability (Section~\ref{sec:how_vary}). This distinction is crucial for biomedical datasets where spatial patterns exhibit systematic variation across patients and tissue sections, and where borrowing strength across the hierarchy improves estimation efficiency and enables more nuanced biological inference
