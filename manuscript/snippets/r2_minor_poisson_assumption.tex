This allows the expected density of target cells to vary flexibly across space as a function of source cell locations and covariates, enabling the framework to represent complex spatial patterns including clustering, repulsion, and distance-dependent associations. While alternative models exist (e.g., Cox processes, cluster processes, Gibbs processes), the inhomogeneous Poisson process provides the best balance of flexibility, interpretability, and computational feasibility for our application~\citep{baddeley_spatial_2015}
