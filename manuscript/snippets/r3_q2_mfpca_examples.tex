For instance, SHADE identified vasculature as driving increased short-range clustering of CTLs in CLR patients (Figure~\ref{fig:sic_CRC_diff}). The $G$-cross mFPCA analysis shows qualitatively similar patterns, with CLR patients exhibiting lower $G$-cross values at short distances (indicating clustering) compared to DII patients. This concordance suggests that SHADE's conditional modeling captures genuine marginal pairwise associations for this interaction. Conversely, for CTL-tumor cell interactions, SHADE detected elevated clustering in DII patients across all distances, while $G$-cross mFPCA curves show more subtle group differences with overlapping uncertainty bands at intermediate distances. This discrepancy may reflect SHADE's multivariate adjustment---controlling for other cell types reveals stronger directional effects than marginal pairwise statistics suggest.
