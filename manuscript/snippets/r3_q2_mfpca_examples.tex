For instance, SHADE identified greater CTL clustering around tumor cells in DII patients across all distances (Figure~\ref{fig:sic_crc_all}), while $G$-cross mFPCA curves show the opposite pattern, with CLR patients exhibiting stronger CTL-tumor associations. This discrepancy likely reflects SHADE's multivariate adjustment: after controlling for other cell types, the conditional CTL-tumor interaction is stronger in DII, whereas marginal pairwise statistics capture different aspects of spatial organization. Similarly, SHADE detected greater granulocyte-TAM clustering in DII patients, but $G$-cross mFPCA shows minimal group differences for this interaction, suggesting that SHADE's hierarchical modeling and conditional framework reveals patterns that are subtle or absent in marginal analyses. Additionally, SHADE identified stronger CTL repulsion from CAFs at medium-long range in CLR patients, a pattern not evident in marginal $G$-cross analyses. These discrepancies highlight how multivariate adjustment and hierarchical pooling can reveal conditional dependencies that differ from marginal pairwise associations.
