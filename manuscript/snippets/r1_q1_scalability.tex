Feature construction for each focal cell type $B$ involves evaluating inter-cell distances between observed and dummy focal locations and all non-focal cells. This step scales as $\mathcal{O}(n_{\mathrm{focal}} \times n_{\mathrm{source}})$, which is quadratic in the total number of cells when the focal and source sets are of similar size. However, this operation is implemented using the optimized \texttt{crossdist} routine from \texttt{spatstat.geom}, which efficiently computes pairwise distances in compiled code. The resulting distance matrix is reused across all basis functions $\{\phi_p\}$ and source types $\{A_k\}$, so the dominant cost occurs only once per focal type.

To assess the practical runtime implications, we performed a timing experiment varying the total number of cells from 5,000 to 250,000 using variational inference (Supplement, Section~\ref{sec:timing_experiments}). Feature construction time scaled as $O(n^{1.46})$ (empirical exponent from log-log regression), while total model fitting time scaled as $O(n^{0.85})$ due to efficient distance matrix reuse. At 100,000 cells, total fitting time was approximately 36 seconds; at 250,000 cells, approximately 133 seconds. These benchmarks demonstrate that SHADE remains computationally tractable for large-scale multiplexed imaging studies.
