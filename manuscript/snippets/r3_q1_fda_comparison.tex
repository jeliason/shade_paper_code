SHADE directly models the underlying point process through the conditional intensity $\lambda(v \mid X_{A_1}, \ldots, X_{A_K})$ (Equation~\ref{eq:cond_int_simplified}), leveraging the point-level likelihood for inference. In contrast, functional data analysis (FDA) approaches applied to spatial statistics (e.g., mFPCA of $G$-cross or $K$-cross functions) first compute per-image summary statistics, then analyze the resulting curves in a second stage—a two-stage procedure that does not propagate uncertainty from point pattern estimation into functional inference. While both methods can leverage hierarchical structure, SHADE's generative modeling framework directly integrates the spatial point process likelihood, enabling simultaneous estimation of spatial interactions and variance components across biological scales through partial pooling.