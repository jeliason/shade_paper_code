Our modeling framework captures \emph{predictive} spatial associations, not biological causation. A strong \( A \to B \) SIC indicates that the presence of \( A \) is statistically predictive of local \( B \) density, meaning that the spatial locations of type $A$ cells provide information about where type $B$ cells are likely to be found. Formally, this means that the conditional intensity function $\lambda(v \mid X_A)$ differs from the marginal (unconditional) intensity $\lambda_B(v)$, which represents the expected density of type $B$ cells in the absence of any spatial dependence on type $A$ locations. Under complete spatial independence, knowing where type $A$ cells are located provides no information about type $B$ density, so $\lambda(v \mid X_A) = \lambda_B(v)$.

Importantly, while our model quantifies the strength and direction of spatial predictability, it does not establish that \( A \) exerts a causal biological effect on \( B \)
