We also tested SHADE's performance when the model is misspecified due to unmeasured spatial heterogeneity—specifically, discrete tissue compartments (e.g., tumor islands, stromal regions) that create baseline density differences independent of source-target interactions (Figure~\ref{fig:compartment_robustness}; \secref{ssec:compartment_confounder}). Results reveal regime-dependent bias: when both cell types are abundant, SHADE achieves perfect detection power but exhibits elevated Type I error rates (11.7--17.1\%) and severely undercovers (43--52\% vs.\ expected 95\%), incorrectly attributing compartment effects to source-target interactions. When target density is low, wider credible bands provide partial robustness (82--93\% coverage, 1.7--5.8\% Type I error). These findings indicate that unmeasured spatial structure can produce substantial bias in high-density scenarios, suggesting the need for explicit compartment modeling or sensitivity analyses when such heterogeneity is suspected.
