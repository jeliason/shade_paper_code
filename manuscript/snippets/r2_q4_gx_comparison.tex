The $G$-cross-based analysis (Figure~\ref{fig:gx_comparison}) reveals several patterns of cell--cell clustering between CLR and DII patients. For instance, TAM--granulocyte interactions showed strong, statistically significant differences at all examined distances (20--60 microns), with higher clustering in DII tumors. SHADE's directional analysis shows concordant results, with DII patients exhibiting greater attraction of granulocytes to TAMs across all distances, suggesting that this marginal clustering pattern reflects genuine directional spatial association. However, $G$-cross and SHADE diverge for other interactions. SHADE identified stronger CTL-tumor clustering in DII patients and differential stromal interactions (e.g., CTL repulsion from CAFs being stronger in CLR at medium-long range), patterns that are less evident or show different magnitudes in the $G$-cross analysis. These discrepancies may reflect SHADE's multivariate adjustment for confounding cell types and hierarchical modeling structure, which can reveal conditional dependencies that differ from marginal pairwise associations captured by $G$-cross.
