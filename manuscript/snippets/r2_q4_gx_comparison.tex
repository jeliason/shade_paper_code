The $G$-cross-based analysis (Figure~\ref{fig:gx_comparison}) reveals several patterns of cell--cell clustering between CLR and DII patients, but notable differences emerge when we compare these results with SHADE-derived SICs. For instance, TAM--granulocyte interactions showed strong, statistically significant differences at all examined distances (20--60 microns), with higher clustering in DII tumors. This suggests a prominent granulocyte-TAM association in DII tumors according to $G$-cross. However, SHADE did not highlight this pair as a major differential interaction between groups, indicating that the clustering of granulocytes around TAMs may not translate into strong directional interactions when adjusting for other cell types or when modeled hierarchically. Conversely, SHADE identified vasculature as a key source driving differences in clustering of both CTLs and memory CD4+ T cells between CLR and DII groups, reflecting enhanced immune surveillance or trafficking around blood vessels in CLR tumors. This vasculature effect is less evident in the $G$-cross analysis, where vasculature-related interactions showed limited significance and mostly small effect sizes.
