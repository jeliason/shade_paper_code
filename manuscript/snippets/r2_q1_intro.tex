Most spatial analysis pipelines for multiplex or histological imaging rely on
standard second-order summary statistics---such as Ripley’s $K$- and
$L$-functions or the $G$-cross function---computed independently for each
image. These per-image summaries (e.g., $K$-function curves or $G$-cross
area-under-curve scores) are then compared across experimental groups or used as
sample-level covariates in downstream models
\citep{baruaSpatialInteractionTumor2018,vuSPFSpatialFunctional2022,caneteSpicyRSpatialAnalysis2022,samorodnitskySpatialOmnibusTest2024,sealSpaceANOVASpatialCooccurrence2024,jingQuantifyingInterpretingBiologically2025,janeiroSpatiallyResolvedTissue2024}.
Although this strategy is simple and widely adopted, it implicitly treats each
image as an isolated point pattern and cannot share information across images or
subjects.

Consequently, such workflows cannot perform partial pooling or decompose
heterogeneity across biological levels such as images, patients, and cohorts.
While hierarchical models for replicated point patterns have been proposed
\citep{bagchiMethodAnalysingReplicated2015a,leeCharacterizingCrosssubjectSpatial2017,myllymakiHierarchicalSecondorderAnalysis2014,bellMixedModelsAnalysis2004,illianGibbsPointProcess2010},
these approaches either apply hierarchical regression to spatial summary statistics (e.g., K-function), use two-stage procedures that separately estimate per-replicate patterns before modeling cross-replicate variation, or fit parametric Gibbs models via maximum pseudolikelihood with random effects, rather than fully Bayesian joint estimation of flexible interaction functions across multiple biological scales. Summary-based methods cannot model conditional interactions at the point process level, while pseudolikelihood approaches with mixed effects impose parametric interaction forms and rely on frequentist approximations rather than full Bayesian posterior inference. This leaves a methodological gap: no existing framework jointly estimates flexible, interpretable spatial interaction curves while borrowing strength across images, patients, and cohorts through multilevel Bayesian inference.
