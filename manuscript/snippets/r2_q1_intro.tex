Standard pipelines for analyzing spatial cellular interactions in multiplexed imaging data typically compute second-order summary statistics, such as Ripley's $K$- and $L$-functions or the $G$-cross function, independently for each image to quantify tumor–immune interactions, infiltration patterns, or neighborhood structure \citep{baruaSpatialInteractionTumor2018,vuSPFSpatialFunctional2022,caneteSpicyRSpatialAnalysis2022,samorodnitskySpatialOmnibusTest2024,sealSpaceANOVASpatialCooccurrence2024,jingQuantifyingInterpretingBiologically2025,janeiroSpatiallyResolvedTissue2024,soupirBenchmarkingSpatialCoLocalization2025}. These per-image summaries are then compared across groups or used as covariates in downstream analyses. Although effective for descriptive comparisons, this approach treats each image as an isolated point pattern and cannot share information across biological replicates during estimation. Hierarchical models for replicated point patterns have been proposed \citep{bagchiMethodAnalysingReplicated2015a,leeCharacterizingCrosssubjectSpatial2017,myllymakiHierarchicalSecondorderAnalysis2014,bellMixedModelsAnalysis2004,illianGibbsPointProcess2010,wrobelMxfdaComprehensiveToolkit2024}, but they typically model derived summaries via two-stage estimation or use parametric models fit via pseudolikelihood with random effects, rather than jointly estimating flexible interaction functions through full Bayesian multilevel inference at the point-process level.
