Comparing heterogeneity patterns between CLR and DII tumor subtypes reveals subtype-specific differences in variability (Supplement Figure~\ref{fig:mad_clr_dii}). At the patient level, DII tumors show greater between-patient heterogeneity than CLR tumors for several immune-related interactions, most notably memory CD4+ T cells with CAFs (MAD difference = 0.015), CTLs with tumor cells (0.014), and granulocytes with vasculature (0.014). Conversely, CLR tumors exhibit greater between-patient variability for granulocyte-CAF interactions (MAD difference = $-0.025$) and memory CD4+ T-TAM interactions ($-0.017$). At the image level, DII tumors show more within-patient heterogeneity for granulocyte-vasculature (MAD difference = 0.028) and memory CD4+ T-tumor interactions (0.017), while CLR tumors show greater within-patient variability for granulocyte interactions with hybrid E/M cells ($-0.034$) and CAFs ($-0.031$), and for CTL-CAF interactions ($-0.020$). These results demonstrate that tumor subtype affects not only the mean spatial interaction patterns (Section~\ref{sec:hot_cold}) but also their variability at both between-patient and within-patient scales. The particularly strong subtype-specific differences in granulocyte-related heterogeneity may reflect distinct modes of myeloid cell recruitment and spatial organization between the two tumor subtypes.
