To provide additional context for SHADE's results and explore complementary perspectives on group-level spatial organization, we compared SHADE's group-level SICs with functional data analysis (FDA) of marginal pairwise summary statistics. Specifically, we applied multilevel functional principal component analysis (mFPCA)~\citep{wrobelMxfdaComprehensiveToolkit2024} separately to CLR and DII patient groups, analyzing $G$-cross and $L$-cross functions computed for each tissue section.

Unlike SHADE, which models conditional intensities adjusting for all cell types simultaneously within a generative point process framework, mFPCA decomposes functional variation in observed summary statistics across hierarchical levels (images nested within patients). This approach does not provide a probabilistic model of the underlying point process, but instead characterizes dominant modes of variation in the empirical $G$-cross and $L$-cross curves. The resulting group-level mean functions with variability bands (mean $\pm$ 1 SD from the first functional principal component) provide a complementary view of pairwise spatial associations. Note that these bands represent descriptive variation captured by the first PC and do not have coverage properties analogous to confidence or credible intervals.
